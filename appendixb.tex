\chapter{付録A 考案したNMFの制約}
本節では,データの特徴を考慮して考案したNMFの制約とモデルエビデンスの計算方法を紹介する.

% \section{分解能の決定}
% ニューロンの活動データの扱いには時間分解能と空間分解能の2つの側面から検討する必要がある.
% 時間分解能については,蛍光強度データをそのまま用いる,時間窓に区切るなどが考えられる.
% 空間分解能については,ニューロン1個を見る場合,2個を見る場合,複数を見る場合が考えられる.
% 手法によってどのレベルでデータを扱うかが異なる.
% \Tabref{tab:methods}にカルシウムイメージングデータを解析する際に使えそうな手法を載せる.
% これらの手法ではニューロンのネットワーク構造変化やグループの活動の変化などを観察できる.
% 提案アプローチでは行列分解を使いニューロングループを推定し,グループの活動時系列の情報は使わなかった.
% しかし,提案アプローチでも時間窓で区切ってニューロングループの推定を行えばクラスタの変化は抽出できる.
% その際,あまりに短い時間窓内では意味のある情報が取り出せない可能性がある.
% ニューロングループがどれくらいの頻度で活動するのかをある程度の仮定を置いて時間窓を決定する必要がある.
% 
% \begin{table}[htb]
%   \center
%   \begin{tabular}{|c|cc|} \hline
%     & 生データ & 時間窓で区切る \\ \hline
%     ペアで見る & 時系列クラスタリング,TE & glasso,類似度+クラスタリング\\
% 	  複数で見る & 行列分解 & ロジスティック回帰,時系列クラスタリング \\ \hline
%   \end{tabular}
%   \caption{カルシウムイメージングデータ解析に使えそうな手法}
%   \label{tab:methods}
% \end{table}
% 
\section{スケール除去}
ニューロンごとに発現している蛍光タンパク質の量や細胞の大きさが異なる.
そのため,ニューロンごとのバイアスが観測データに載っていると考え,バイアスを除去する方法を試した.
この時の数理モデルは以下のようになる:
\begin{equation}
	X = DC + H + B.
\end{equation}
ただし,$B \in \mathbb{R}_+^{I\times J}$は行ごとに同じ数値が入ったバイアス行列である.

バイアスの推定方法は,$D$に1列を足し,$C$に$\mathbf 1$の1行を足してNMFを更新する.
$D$の列にバイアスが推定されることを期待した.

簡単な人工データ実験を行った結果,足したバイアスよりも大きいバイアスが推定されてしまうことがわかった.
また,そもそもの数理モデルが異なると考え直した.

蛍光タンパク質の量や細胞の大きさが異なるということは,各ニューロンはスケールされていると考えられ,以下のように表される:
\begin{equation}
	X = A(DC + H).
\end{equation}
ただし,$A \in \mathbb{R}_+^{I \times I}$は対角行列である.

ナイーブな求め方は,$A$は単位行列で初期化し,NMF一回の更新ごとに$X$をニューロンごとに残差の四分位範囲で割り,その値を$A$にかけていく.
簡単な人工データ実験で乗法更新則を用いた場合は$A$は真の値に近いものが推定された.

\section{重複除去}
本研究で置いた仮定では,あるニューロンが複数のグループに所属する時,グループの活動は被らないとしている.
しかし,NMFの推定時にそのような制約は入れていないので,NMFで推定した結果この仮定が破られているようであれば制約は入れなければならない.

そこで,以下の目的関数を考えた:
\begin{align}
	\argmin_{D \geq 0, C \geq 0} ||X - DC||_F^2 - \lambda \sum_{k=1}^K \sum_{l \neq k}^K \left( || d_{:l} - d_{:k} ||_1 || c_{l:} - c_{k:} ||_1 \right).
  \label{eq:overlap}
\end{align}

更新則を導出する.
参考にしたのは\cite{Babaee2016}である.
\eqref{eq:overlap}のLagrange関数$L$は,
\begin{align}
	L = &\text{Tr}(X^TX) - 2 \text{Tr}(X^TDC) + \text{Tr}(C^TD^TDC) - \text{Tr}(\Phi_C C^T)\\
	&- \text{Tr}(\Phi_D D^T) - \lambda \text{Tr} (F^T C H^T S^T D F),
\end{align}
であり,KKT条件は,
\begin{align}
	\frac{\partial L}{\partial C} = \frac{\partial L}{\partial D} = 0 \\
	D \geq 0 \\
	C \geq 0 \\
	\Phi_C \geq 0 \\
	\Phi_D \geq 0 \\
	\Phi_C C = \Phi_D D = 0
\end{align}
である.
ただし,$\Phi_D$と$\Phi_C$はそれぞれ$D \geq 0$,$C \geq 0$に対するLagrange乗数で,$F \in [0,1]^{K \times (K-1)!}$は2つの時間の組み合わせを表現した以下のような行列である:
\begin{equation}
	F = \left(
    \begin{array}{cccccc}
			1 & 1 & \ldots & 0 & \ldots & 0 \\
			-1 & 0 & \ldots & 1 & \ldots & 0 \\
			0 & -1 & \ldots & -1 & \ldots & 0 \\
			\vdots & \vdots & \vdots & \vdots & \ddots & \vdots \\
			0 & 0 & \ldots & 0 & \ldots & 1 \\
			0 & 0 & \ldots & 0 & \ldots & -1
    \end{array}
  \right).
\end{equation}
また,$S = sign(DF)$,$H = sign(F^TC)$とおく.

$D$を求めるには以下の式を解く:
\begin{equation}
	\frac{\partial L}{\partial D} = - 2 X C^T + 2 DCC^T - \Phi_D - \lambda SHC^T FF^T = 0.
\end{equation}
これを求めると$D$の要素の更新は以下である:
\begin{equation}
	d_{ik} \leftarrow d_{ik} \frac{2[XC^T]_{ik} + \lambda [SHC^T FF^T]_{ik}^+}{2[DCC^T]_{ik} - \lambda [SHC^T FF^T]_{ik}^-}.
\end{equation}
ただし,$[\cdot]^+$は行列の中の正の要素,$[\cdot]^-$は負の要素である.

同様に,$C$の要素の更新は以下の通りである:
\begin{equation}
	c_{kj} \leftarrow c_{kj} \frac{2[D^T X]_{kj} + \lambda [FF^T D^T SH]_{kj}^+}{2[D^T DC]_{kj} - \lambda [FF^T D^T SH]_{kj}^-}.
\end{equation}

しかし,ハイパーパラメータ$\lambda$の最適な決め方は分からない.
また,$\lambda$が大きいと$D$と$C$はスパースになるが,本来2つの基底に所属するニューロンの$d_{i:}$が1つの基底以外0に近くなることが実験を通してわかった.
この手法は改善の余地がある上,実際のデータをNMFで分解した際にどれくらい仮定に沿っていないかを確認した上で使う必要がある.

\section{時間方向への制約}
カルシウムイメージングデータはスパイク情報を反映するのが遅く,一度上がった蛍光強度は緩やかに下がっていく.
そのため,NMFで分解を行う際も,行列$C$の時間方向に前時刻の値と近くなるような制約を入れることでより正確なニューロングループの抽出が行えると考えられる.
$C$の偶数列を前後の列の平均とするNMFも提案されている~\cite{Cheung2015}.
しかし,これはかなりスムースになる制約だと考えられる.
そこで,以下のような制約を加えた目的関数が考えられる:
\begin{equation}
	\argmin_{D \geq 0, C \geq 0}||X - DC||^2_F + \lambda \sum_t||c_{:t} - c_{:t-1}||_1.
  \label{eq:NMF-ar}
\end{equation}
これはfused lasso \cite{Tibshirani2005}と同じような制約である.

更新則は,$D$は\eqref{eq:nmf_fro}と同じだが$C$は異なり以下である:
\begin{equation}
	c_{kj} \leftarrow c_{kj} \frac{2[D^T X]_{kj} - s_{kj}}{2[D^T DC]_{kj}}.
\end{equation}
ただし,$s_{:j} = sign(c_{:j} - c_{:j-1})$であり,1列目のみ$s_{:1} = \mathbf{0}$である.

重複除去制約と同じく,$\lambda$の決め方が分からないため使用は断念した.
NMFは初期値依存の問題があるため,$\lambda$を変化させると解空間も変化し,結果は滑らかに変化しない.
そのため,$\lambda$を変化させた効果を比較するのは難しい.
