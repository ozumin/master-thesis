\chapter{人工データ実験}
\section{Simulation}
\subsection{Spiking model}
To evaluate how accurately NMF can obtain neural ensembles, we conducted an experiment on artificial data.

We use spiking model proposed by Izhikevich~\cite{Izhikevich2003}.
This model is based on Hodgkin-Huxley model with computational efficiency.
There are 4 parameters in thid model which characterize neuron types.
We used regular spiking neurons for excitatory neurons and fast spiking neurons for inhibitory neurons.
The parameters we used is shown in \Tabref{tab:parameter1} where $r_e$ and $r_i$ are random variables following a uniform distribution from 0 to 1.

\begin{table}[htb]
  \center
  \begin{tabular}{|c|cccc|} \hline
    neuron type & a & b & c & d \\ \hline
    excitatory neuron & 0.02 & 0.2 & $-65 + 15 r_e^2$ & $8 - 6r_e^2$ \\
    inhibitory neuron & $0.02 + 0.08r_i$ & $0.25 - 0.05 r_i$ & -65 & 2 \\ \hline
  \end{tabular}
  \caption{Parameters used in Izhikevich model}
  \label{tab:parameter1}
\end{table}

We need to specify synaptic transmissions between neurons.
It can be written by adjacency matrix.
The $(i,j)$-element of the adjacency matrix equals to how much voltage will be transmitted from neuron $j$ to neuron $i$ when neuron $j$ is fired.
The columns of excitatory neurons and inhibitory neurons are random variables following a uniform distribution from 0 to 0.5 and a uniform distribution from -2 to 0, respectively.

We simulated for a network of 800 excitatory neurons and 200 inhibitory neurons.
Every neuron has random thalamic input from outside the network in every 1ms.
Thalamic input for excitatory neurons and inhibitory neurons follow a gaussian distribution with mean 0 and variance 5 and a gaussian distribution with mean 0 and variance 2, respectively.
We need to evaluate NMF by the ability of detecting neural ensembles which is activated in specific time.
The number of neurons in each ensemble follows a uniform distribution from 50 to 200.
We changed neural ensembles every 10s.
In that time period, neurons in ensemble have stronger thalamic input raised by 1 and 0.4 for excitatory and inhibitory neurons, respectively.
Total time of simulation is 470s.
The first 10s is not used for the simulation stability.

\subsection{Calcium imaging model}
We calculate calcium concentration in a neuron from simulated spikes by following model proposed by Vogelstein~\cite{Vogelstein2009}:
\begin{equation}
  [Ca^{2+}]_{i,t} - [Ca^{2+}]_{i,t-1} = - \frac{\Delta}{\tau}([Ca^{2+}]_{t-1} - [Ca^{2+}]_b) + An_{i,t} + \sigma_c \sqrt{\Delta} \epsilon_{i,t},
  \label{eq:calcium}
\end{equation}
where $[Ca^{2+}]_{i,t}$ is calcium concentration of neuron $i$ at time $t$, $[Ca^{2+}]_b$ is calcium concentration baseline, $\Delta$ is time step size, $\tau$ is decay time constant, $A$ is rise in $[Ca^{2+}]$ after spike, $n_{i,t}$ is spike of 0 or 1, $\sigma_c$ is variance of noise, $\epsilon_{i,t}$ is noise which follows normal gaussian distribution.
We do not take saturation into account because our data seems to be not having saturation.

Then, we convert the calcium concentration $[Ca^{2+}]_{i,t}$ to imaging intensity $F_{i,t}$ by equation from Vogelstein~\cite{Vogelstein2009}:
\begin{equation}
  F_{i,t} = \alpha[Ca^{2+}]_t + \beta + \sigma_F \epsilon_{i,t},
  \label{eq:intensity}
\end{equation}
where $\alpha$ is the scale of imaging intensity, $\beta$ is offset, $sigma_F$ is variance of noise and $\epsilon_{i,t}$ is noise which follows normal gaussian distribution.

Finally, we take sum of intensity every 125ms since our data is sampled at 8Hz:
\begin{equation}
  x_{i,t'} = \sum_{t=1}^{125} F_{i,t}.
  \label{eq:observation}
\end{equation}

Parameters we used is shown in \Tabref{tab:parameter2}.
\begin{table}[htb]
  \center
  \begin{tabular}{|cccccccc|} \hline
    $[Ca^{2+}]_b$ & $\Delta$ & $\tau$ & $A$ & $\sigma_c$ & $\alpha$ & $\beta$ & $\sigma_F$ \\ \hline
    0.1 & 0.001 & 0.5 & 5.0 & 1.0 & 1.0 & 0 & 1.0 \\ \hline
  \end{tabular}
  \caption{Parameters used in Vogelstein model}
  \label{tab:parameter2}
\end{table}

\section{Result}
