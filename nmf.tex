\section{NMFでのアプローチ}
\subsection{データに対するアプローチ}
\subsubsection{バイアス除去}
やってみたが割とバイアスに持っていかれる.
そもそも数理モデルが違いそう.

\subsubsection{重複除去}
やってみたがうまくいかんかった.

\subsubsection{時間方向への制約}
カルシウムイメージングデータはスパイク情報を反映するのが遅く,一度上がった蛍光強度は緩やかに下がっていく.
そのため,NMFで分解を行う際も,行列$C$の時間方向に前時刻の値と近くなるような制約を入れることでより正確なニューロングループの抽出が行えると考えられる.
そこで,以下のような制約を加えた目的関数が考えられる:
\begin{equation}
	\argmin_{D \geq 0, C \geq 0}||{\mb X - DC}||^2_F + \lambda \sum_t||{\mb C}[:,t] - {\mb C}[:,t-1]||_1.
  \label{eq:NMF-ar}
\end{equation}

\subsection{ブートストラップ}
これで基底決めるのから逃れられる..
