\chapter{手法}
\section{仮説}
We made several assumptions before making mathematical model for calcium imaging data.
\\
\noindent \textbf{Assumption 1}\\
There are $K$ neuronal ensembles, whose members become active simultaneously.
A neuron can be in several ensembles.
Members in each ensemble do not change during observation.\\
\textbf{Assumption 2}\\
When several ensembles are activated, their members do not overlap.
\\
With these assumptions, we can construct a mathematical model.
Let $\boldsymbol{c}_{k:}$, $k=1,\dots,K$ be a time series of the activity of neuronal ensemble $k$.
For neuron $i$, $\boldsymbol c$ is weighted to construct the activity of neuron $i$.
Let $\boldsymbol{y}^i_{k:}$, ($k = 1, \dots, K$) be a weighted time series of $\boldsymbol{c}_{k:}$ written as follows:
\begin{equation}
  \boldsymbol{y}^i_{k:} = d_{ik} \boldsymbol{c}_{k:}.
  \label{eq:y}
\end{equation}

Let $\boldsymbol{x}_{i:}$ be a time series of $i$th neuron of observed data.
The observation $\boldsymbol{x}_{i:}$ is modeled as follows:
\begin{equation}
  \boldsymbol{x}_{i:} = \sum_{k=1}^K \boldsymbol{y}^i_{k:} + \boldsymbol{\eta}_{i:}.
  \label{eq:x}
\end{equation}
$\boldsymbol{\eta}_{i:}$ is a time series of noise which is normally distributed.
The noise of calcium imaging is mostly photon shot noise, which obeys a Poisson distribution.
For high photon counts, shot noise can be approximated by a Gaussian distribution~\cite{Sjulson2007}.

Let $\mathbb{R}_+$ be a set of non-negative real numbers.
This model can be expressed in a matrix form as follows:
\begin{eqnarray}
  \boldsymbol{Y} &=& \boldsymbol{DC}, \\
  \boldsymbol{X} &=& \boldsymbol{Y} + \boldsymbol{H},
  \label{model_matrix}
\end{eqnarray}
where $\boldsymbol{X} \in \mathbb{R}_+^{N \times T}$, $\boldsymbol{D} \in \mathbb{R}_+^{N \times K}$, $\boldsymbol{C} \in \mathbb{R}_+^{K \times T}$, and $\boldsymbol{H} \in \mathbb{R}^{N \times T}$.
The row of $\boldsymbol X$ is $\boldsymbol{x}_{i:}$, the $(i,k)$-element of $\boldsymbol D$ is $d_{ik}$, the row of $\boldsymbol C$ is $\boldsymbol{c}_{i:}$, and the row of $\boldsymbol H$ is $\boldsymbol{\eta}_{i:}$.

Then, the problem is to estimate $\boldsymbol D$ and $\boldsymbol C$.

We can use non-negative matrix factorzation (NMF) to estimate $\boldsymbol D$ and $\boldsymbol C$.
NMF decomposes a non-negative matrix $\boldsymbol{X}$ into a product of non-negative matrices $\boldsymbol{D}$ and $\boldsymbol{C}$.
The variances of noise in each neuron differs because $\boldsymbol A$ is multiplied to the noise matrix $\boldsymbol H$.
NMF which use euclidean distance in objective function cannot consider different noise variances into account.

We define a contribution rate of $\boldsymbol{y}^i_{k:}$ for neuron $i$ by its l1-norm:
\begin{equation}
  || \boldsymbol{y}^i_{k:}||_1.
  \label{eq:contribution}
\end{equation}
Our goal is to obtain neural ensembels; however, the result of NMF is not clear which neural ensembles neuron $i$ belongs to.
If a neuron belongs to a neural ensemble $k$, its contribution $||\boldsymbol{y}^i_{k:}||_1$ will be large.
Note that we should devide the contribution rate by $\sum_{k=1}^K ||\boldsymbol{y}^i_{k:}||_1$ to compare contribution between neurons.

\section{解析のアプローチ}
\subsection{分解能の決定}
ニューロンの活動データの扱いには時間分解能と空間分解能の2つの側面から検討する必要がある.
時間分解能については,蛍光強度データをそのまま用いる場合,移動平均をとる場合,時間窓に区切るなどが考えられる.
空間分解能については,ニューロン1個を見る場合,2個を見る場合,グループとして見る場合が考えられる.
手法によってどのレベルでデータを扱うかが異なる.

\subsection{Graphical lasso}

\subsection{ロジスティック回帰}

\subsection{NMF}
Nonnegative matrix factorization(NMF)は行列分解の手法の一つである.
NMFは以下の目的関数を最小化する:
\begin{equation}
	\argmin_{D \geq 0, C \geq 0} ||X - DC||_F^2.
  \label{eq:NMF}
\end{equation}
NMFで検討したアプローチは次章で説明する.
