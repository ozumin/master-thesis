\chapter{手法}
本章では検討した解析方法について述べる.
\section{解析のアプローチ}
\subsection{分解能の決定}
ニューロンの活動データの扱いには時間分解能と空間分解能の2つの側面から検討する必要がある.
時間分解能については,蛍光強度データをそのまま用いる,移動平均をとる,時間窓に区切るなどが考えられる.
空間分解能については,ニューロン1個を見る場合,2個を見る場合,複数を見る場合が考えられる.
手法によってどのレベルでデータを扱うかが異なる.
\Tabref{tab:methods}にカルシウムイメージングデータを解析する際に使えそうな手法を載せる.

\begin{table}[htb]
  \center
  \begin{tabular}{|c|cc|} \hline
    & 生データ & 時間窓で区切る \\ \hline
    ペアで見る & 時系列クラスタリング & glasso,類似度+クラスタリング\\
	  複数で見る & 行列分解 & ロジスティック回帰,時系列クラスタリング \\ \hline
  \end{tabular}
  \caption{カルシウムイメージングデータ解析に使えそうな手法}
  \label{tab:methods}
\end{table}

本論文では,データに対して仮説をおいて数理モデルを立てた上で,行列分解を用いることにする.

\section{モデル}
カルシウムイメージングデータを解析するに当たって,いくつかの仮説をおいた.
\\
\noindent \textbf{仮説 1}\\
グループが$K$個存在し,同じグループのニューロンは同時に活動する.
ニューロンは複数のグループに所属することができる.
観測時間内ではグループに属するニューロンは変化しない.\\
\textbf{仮説 2}\\
複数のグループが同時に活動する時,属するニューロンは被らない(ニューロンが属するグループは同時には活動しない).
\\

これらの仮説をもとに,数理モデルを構築する.
$c_k$, ($k=1,\dots,K$)をグループ$k$の活動の時系列とする.

ニューロン$i$の観測時系列を$x_i$とおき,$c_k$の重み付け和で表す:
\begin{equation}
  x_i = \sum_{k=1}^K d_{ik} c_k + \eta_i.
  \label{eq:x}
\end{equation}
$\eta_i$はガウスノイズである.
カルシウムイメージングのノイズはポアソン分布に従う光子ノイズであるが,光子数が多い場合はガウス分布で近似できる~\cite{Sjulson2007}.

$\mathbb{R}_+$を非負の実数の集合とする.
\Eqref{eq:x}は行列形式で以下のように表現できる:
\begin{eqnarray}
  Y &=& DC, \\
  X &=& Y + H.
  \label{model_matrix}
\end{eqnarray}
ただし,$X \in \mathbb{R}_+^{N \times T}$, $D \in \mathbb{R}_+^{N \times K}$, $C \in \mathbb{R}_+^{K \times T}$, and $H \in \mathbb{R}^{N \times T}$である.
また,$X$の行は$x_i$,$D$の要素$(i,k)$は$d_{ik}$,$C$の行は$c_i$,$H$の行は$\eta_i$である.
これ以降,行列$A$の$i$行を$A_{i:}$,$j$列を$A_{:j}$,$(i,j)$要素を$A_{ij}$と表記する.

解くべき問題は$D$と$C$を推定することである.

ニューロンがどのグループに所属しているかを評価する方法として,寄与率を導入する.
ニューロン$i$に対するループ$k$の寄与率を以下のように定義する:
\begin{equation}
	\frac{|| d_{ik} c_k ||_1}{\sum_{l=1}^K || d_{il} c_l ||_1}.
  \label{eq:pov}
\end{equation}
寄与率が高いグループにニューロン$i$が所属すると考えることにする.
