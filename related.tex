\section{関連研究}
\subsection{脳データの解析}
カルシウムイメージングは時間解像度が低く空間解像度が高いデータが取得できる手法である.
同じ特徴であるfMRIに対するデータの解析手法を紹介する.

fMRIデータの解析はmodel basedな手法とdata-drivenな手法がある\cite{Li2009}.
Model basedな手法の例はstatistical parametric mapping (SPM)やcross-correlation analysis (CCA),coherence analysis (CA)などが上げられる.
Data-drivenな手法は更にdecompositionとclusteringに分けることができる.
Decompositionにはprincinpal component analysis (PCA)やsingular value decomposition (SVD),independent component analysis (ICA)などが挙げられる.
Clusteringはfuzzy clustering analysis (FCA)やhierarchical clustering analysis (HCA)などがある.

\subsection{カルシウムイメージングの解析例}
カルシウムイメージングデータの解析には二種類考えられる.
まずは,カルシウムイメージングデータからスパイクを推定し,そのスパイク列を解析する方法である.
Vogelsteinらは逐次モンテカルロ法を用いてカルシウムイメージングデータからスパイク推定を行なった\cite{Vogelstein2009}.
しかし,カルシウムイメージングデータでも低いサンプリングレートで計測されたデータではこの方法は使えない.

もう一つは,データから直接ニューロンの活動を解析する方法である.
Mishchenckoはベイズ推定を用いたニューロンの結合推定を行なった\cite{Mishchencko2011}.
しかし,カルシウムイメージングのサンプリングレートが30Hz以上でないと意味のある結合推定結果は得られないと報告している.
また,StetterらはTransfer Entropyを用いて培養された興奮性ニューロンの結合推定を行なった\cite{Stetter2012}.
この手法では,モデルを仮定せずにデータからネットワーク推定を行なっている.
また,ニューロンのネットワーク構造を仮定した人工のカルシウムイメージングデータを作成し,推定精度を議論している.
Ikegayaらは蛍光強度データの一次微分から発火のタイミングの情報を取り出し,テンプレートマッチングによってリアクチベーション現象を解析した\cite{Ikegaya2004}.

Molterらはカルシウムイメージングデータからニューロングループを抽出する方法を8つ人工データ実験と共に試した\cite{Molter2018}.
手法は大きく2つに分けられ,ニューロンペアの相関を見るものと,全てのニューロン活動の状態を見るものである.
前者ではPCAによって相互相関行列を作成した後,ICAやPromax rotationによってグルーピングを行う.後者では,ニューロンの活動からSVD,k-meansラスタリング,spectralクラスタリングなどを用いてグルーピングを行う.後者の方法では,各グループの時間方向の活動を平均をとるなどして,グループの活動としていた.
人工データは,ニューロンをポアソン分布にしたがって発火させ,発火からカルシウムイメージングの観測データに変換していた.
同じグループに所属するニューロンは発火確率を同じ時間帯にあげることで表現していた.

Ghandorらは学習に関係するニューロン(engram cell)群についてNMFを用いて解析を行った\cite{Ghandour2019}.
実験は複数セッションに分けて行われており,それぞれのセッションでNMFを用いてニューロングループとその活動に分解していた.
基底数はAICcで決めていた.
セッションごとに推定されたグループが近いかをcosine similarityで計った結果,engram cellではnon-engram cellよりも繰り返し活動するグループが多かった.
NMFによってニューロングループがどれほど抽出できるかは論じられていなかった.

\section{目的}
カルシウムイメージングデータは上述のように多くのニューロンを観測できる利点がある一方,時間分解能が低いという欠点を持つ.
また,本研究で扱うデータは低いサンプリングレートで観測されたデータである.
これより,本データでは同時に活動するニューロンを解析することが適当である.

本研究の目的は,カルシウムイメージングデータから同時に活動するニューロングループを推定し,それらの睡眠・覚醒時の活動の違いを解析することである.
ニューロングループの推定にはデータに合ったアプローチを模索する.
また,人工データ実験によって,カルシウムイメージングデータからニューロングループの情報がどの程度取り出せるかを確かめる.

本論文の構成は以下の通りである.
\textcolor{red}{ここはあとで}
