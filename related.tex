\section{関連研究}
\subsection{脳データの解析}
カルシウムイメージングは時間解像度が低く空間解像度が高いデータが取得できる手法である.
同じ特徴であるfMRIに対するデータの解析手法を紹介する.

fMRIデータの解析はmodel basedな手法とdata-drivenな手法がある\cite{Li2009}.
Model basedな手法の例はstatistical parametric mapping (SPM)やcross-correlation analysis (CCA),coherence analysis (CA)などが上げられる.
Data-drivenな手法は更にdecompositionとclusteringに分けることができる.
Decompositionにはprincinpal component analysis (PCA)やsingular value decomposition (SVD),independent component analysis (ICA)などが挙げられる.
Clusteringはfuzzy clustering analysis (FCA)やhierarchical clustering analysis (HCA)などがある.

\subsection{カルシウムイメージングの解析例}
カルシウムイメージングデータの解析には二種類考えられる.
まずは,カルシウムイメージングデータからスパイクを推定し,そのスパイク列を解析する方法である.
Vogelsteinらは逐次モンテカルロ法を用いてカルシウムイメージングデータからスパイク推定を行なった\cite{Vogelstein2009}.
しかし,カルシウムイメージングデータでも低いサンプリングレートで計測されたデータではこの方法は使えない.

もう一つは,データから直接ニューロンの活動を解析する方法である.
Mishchenckoはベイズ推定を用いたニューロンの結合推定を行なった\cite{Mishchencko2011}.
しかし,カルシウムイメージングのサンプリングレートが30Hz以上でないと意味のある結合推定結果は得られないと報告している.
また,StetterらはTransfer Entropyを用いて培養された興奮性ニューロンの結合推定を行なった\cite{Stetter2012}.
この手法では,モデルを仮定せずにデータからネットワーク推定を行なっている.
また,ニューロンのネットワーク構造を仮定した人工のカルシウムイメージングデータを作成し,推定精度を議論している.
Ikegayaらは蛍光強度データの一次微分から発火のタイミングの情報を取り出し,テンプレートマッチングによってリアクチベーション現象を解析した\cite{Ikegaya2004}.

Molterらはカルシウムイメージングデータからニューロングループを抽出する方法を8つ人工データ実験と共に試した\cite{Molter2018}.
手法は大きく2つに分けられ,ニューロンペアの相関を見るものと,全てのニューロン活動の状態を見るものである.
前者では固有値分解によって相関行列を作成した後,ICAやPromax rotationによってグルーピングを行う.後者では,ニューロンの活動からSVD,k-meansラスタリング,spectralクラスタリングなどを用いてグルーピングを行う.後者の方法では,各グループの時間方向の活動を平均をとるなどして,グループの活動としていた.
人工データは,ニューロンをポアソン分布にしたがって発火させ,発火からカルシウムイメージングの観測データに変換していた.
同じグループに所属するニューロンは発火確率を同じ時間帯にあげることで表現していた.
ニューロンは2つのグループに所属する場合も考えられていた.
最も良いと結論づけられていた手法はICA-CSとSGCという手法だったが,安定して推定精度が高くなるのは観測時間が1800sより長い場合だった.

Ghandorらは学習に関係するニューロン(engram cell)群についてNMFを用いて解析を行った\cite{Ghandour2019}.
実験は複数セッションに分けて行われており,それぞれのセッションでNMFを用いてニューロングループとその活動に分解していた.
基底数はAICcで決めていた.
セッションごとに推定されたグループが近いかをcosine similarityで計った結果,engram cellではnon-engram cellよりも繰り返し活動するグループが多かった.
NMFによってニューロングループがどれほど抽出できるかは論じられていなかった.

\subsection{分解能の決定}
ニューロンの活動データの扱いには時間分解能と空間分解能の2つの側面から検討する必要がある.
時間分解能については,蛍光強度データをそのまま用いる,時間窓に区切るなどが考えられる.
空間分解能については,ニューロン1個を見る場合,2個を見る場合,複数を見る場合が考えられる.
手法によってどのレベルでデータを扱うかが異なる.
\Tabref{tab:methods}にカルシウムイメージングデータを解析する際に使えそうな手法を載せる.

\begin{table}[htb]
  \center
  \begin{tabular}{|c|cc|} \hline
    & 生データ & 時間窓で区切る \\ \hline
    ペアで見る & 時系列クラスタリング & glasso,類似度+クラスタリング\\
	  複数で見る & 行列分解 & ロジスティック回帰,時系列クラスタリング \\ \hline
  \end{tabular}
  \caption{カルシウムイメージングデータ解析に使えそうな手法}
  \label{tab:methods}
\end{table}

\section{目的}
カルシウムイメージングデータは上述のように多くのニューロンを観測できる利点がある一方,時間分解能が低いという欠点を持つ.
本研究で扱うデータは低いサンプリングレートで観測されたデータである.
ニューロンは複数で活動することで情報伝達を行っており,睡眠時に多数のニューロンが同時に活動する現象も確認されている.
これより,本データでは同時に活動するニューロングループを推定して解析を行うことが適当である.

本研究の目的は,カルシウムイメージングデータから同時に活動するニューロングループを推定し,それらの睡眠・覚醒時の活動の違いを解析することである.
ニューロングループの推定には,ニューロン同士の類似度行列を作成しそれをクラスタリングする方法をとる.
また,人工データ実験によって,本手法によってカルシウムイメージングデータからニューロングループの情報がどの程度取り出せるかを確かめる.

本論文の構成は以下の通りである.
第2章では数理モデルを元にした解析のアプローチを説明する.
第3章では人工データの作成方法と人工データ実験の結果を述べる.
第4章では実データ解析の結果を述べる.
第5章では採用に至らなかった検討事項について述べ,第6章に結論を述べる.
