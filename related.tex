\chapter{関連研究}
\section{脳のデータをどう解析するか}
fMRIなどの時間解像度が低いデータについて.
スパイクなどの時間解像度が高いデータについて.

\section{カルシウムイメージングの解析例について}
Molterらはカルシウムイメージングデータからニューロングループを抽出する方法を8つ人工データ実験と共に試した\cite{Molter2018}.
手法は大きく2つに分けられ,ニューロンペアの相関を見るものと,全てのニューロン活動の状態を見るものである.
前者ではPCAによって相互相関行列を作成した後,ICAやPromax rotationによってグルーピングを行う.後者では,ニューロンの活動からSVD,k-meansラスタリング,spectralクラスタリングなどを用いてグルーピングを行う.後者の方法では,各グループの時間方向の活動を平均をとるなどして,グループの活動としていた.
人工データは,ニューロンをポアソン分布にしたがって発火させ,発火からカルシウムイメージングの観測データに変換していた.
同じグループに所属するニューロンは発火確率を同じ時間帯にあげることで表現したいた.

後富山大のやつ

\section{bayesian model averagingについて}
ここに入れるかは迷い中
