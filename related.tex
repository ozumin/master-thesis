\section{関連研究}
\subsection{脳データの解析}
カルシウムイメージングは時間解像度が低く空間解像度が高いデータが取得できる手法である.
同じ特徴であるfMRIに対するデータの解析手法を紹介する.

fMRIデータの解析はmodel basedな手法とdata-drivenな手法がある\cite{Li2009}.
Model basedな手法の例はstatistical parametric mapping (SPM)やcross-correlation analysis (CCA),coherence analysis (CA)などが上げられる.
Data-drivenな手法は更にdecompositionとclusteringに分けることができる.
Decompositionにはprincinpal component analysis (PCA)やsingular value decomposition (SVD),independent component analysis (ICA)などが挙げられる.
Clusteringはfuzzy clustering analysis (FCA)やhierarchical clustering analysis (HCA)などがある.

\subsection{カルシウムイメージングの解析例について}
Molterらはカルシウムイメージングデータからニューロングループを抽出する方法を8つ人工データ実験と共に試した\cite{Molter2018}.
手法は大きく2つに分けられ,ニューロンペアの相関を見るものと,全てのニューロン活動の状態を見るものである.
前者ではPCAによって相互相関行列を作成した後,ICAやPromax rotationによってグルーピングを行う.後者では,ニューロンの活動からSVD,k-meansラスタリング,spectralクラスタリングなどを用いてグルーピングを行う.後者の方法では,各グループの時間方向の活動を平均をとるなどして,グループの活動としていた.
人工データは,ニューロンをポアソン分布にしたがって発火させ,発火からカルシウムイメージングの観測データに変換していた.
同じグループに所属するニューロンは発火確率を同じ時間帯にあげることで表現したいた.

Ghandorらは学習に関係するニューロン群についてNMFを用いて解析を行った\cite{Ghandour2019}.
実験は複数セッションに分けて行われており,それぞれのセッションでNMFを用いてニューロングループとその活動に分解していた.
基底数はAICで決めていた.
セッションごとに推定されたグループが近いかをcosine similarityではかった結果,engram cellではnon-engram cellよりも繰り返し活動するグループが多かった.
