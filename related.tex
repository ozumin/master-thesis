\section{関連研究}
\subsection{脳データの解析}
カルシウムイメージングは時間解像度が低く空間解像度が高いデータが取得できる手法である.
同じ特徴であるfMRIに対するデータの解析手法を紹介する.

fMRIデータの解析はmodel basedな手法とdata-drivenな手法がある\cite{Li2009}.
Model basedな手法はcross-correlation analysis (CCA)やcoherence analysis (CA)などが挙げられ,各指標に基づいて脳部位(ROI)の機能的な結合(前述のfunctional connectivity)を推定する.
CCAでは2つのROIの相関を使う.
CAでは,2つのROIの周波数ドメインでの相関を見る.
fMRIのBOLD信号の周期は10[s]ほどなので,0.1[Hz]以下の周波数ドメインでの相関を見る.
結合の有無は各指標の閾値で決めなければならない.
Data-drivenな手法は更に行列分解とクラスタリングに分けることができる.
行列分解にはprincinpal component analysis (PCA)やsingular value decomposition (SVD),independent component analysis (ICA)などが挙げられ,BOLD信号を複数の成分に分解することができる.
同じ成分の重みが大きいROIは結合があるとする.
ICAは成分数を事前に与える必要がある.
クラスタリングはfuzzy clustering analysis (FCA)やhierarchical clustering analysis (HCA)などがある.
クラスタリングではデータ間の距離を定義する必要がある.
FCAはあるROIが複数のクラスタに存在できるクラスタリング手法で,クラスタ数を事前に決める必要がある.
HCAはクラスタ数を決めずにROIやクラスタが近いか遠いかを見ることができる.

本研究では,行列分解手法の1つであるNMFを用いて隣接行列を推定する.
数理モデルを仮定すると,カルシウムイメージングデータにはICAやPCAよりNMFの方がデータに適している.

\subsection{カルシウムイメージングの解析例}
カルシウムイメージングデータの解析には二種類考えられる.
まずは,カルシウムイメージングデータからスパイクを推定し,そのスパイク列を解析する方法である.
Vogelsteinらは逐次モンテカルロ法を用いてカルシウムイメージングデータからスパイク推定を行なった\cite{Vogelstein2009}.
しかし,カルシウムイメージングデータでも低いサンプリングレートで計測されたデータではこの方法は使えない.

もう一つは,データから直接ニューロンの活動を解析する方法である.
Mishchenckoはベイズ推定を用いたニューロンの結合推定を行なった\cite{Mishchencko2011}.
しかし,カルシウムイメージングのサンプリングレートが30Hz以上でないと意味のある結合推定結果は得られないと報告している.
また,Stetterらはtransfer entropy(TE)を用いて培養された興奮性ニューロンの結合推定を行なった\cite{Stetter2012}.
この手法では,2つのニューロン間のTEを計算し,向きのあるエッジを推定している.
また,ニューロンのネットワーク構造を仮定した人工のカルシウムイメージングデータを作成し,推定精度を議論している.
しかし,本研究で扱うデータはサンプリングレートが低いため,向きの情報は取り出せない.
Ikegayaらは蛍光強度データの一次微分を閾値で切ることで発火のタイミングの情報を取り出し,テンプレートマッチングによってリアクチベーション現象を解析した\cite{Ikegaya2004}.
この研究では1Hzのデータからもリアクチベーション現象を確認できていた.
しかし,本研究ではニューロン集団を推定したいため,この方法はあまり適さない.

Molterらはカルシウムイメージングデータからニューロングループを抽出する方法を8つ人工データ実験と共に試した\cite{Molter2018}.
手法は大きく2つに分けられ,ニューロンペアの相関を見るものと,全てのニューロン活動の状態を見るものである.
前者では固有値分解によって相関行列を作成した後,ICAやPromax rotationによってグルーピングを行う.
後者では,ニューロンの活動からSVD,k-meansクラスタリング,スペクトラルクラスタリングなどを用いてグルーピングを行う.
後者の方法では,各グループの時間方向の活動を平均をとるなどして,グループの活動としていた.
人工データは,ニューロンをポアソン分布にしたがって発火させ,発火からカルシウムイメージングの観測データに変換していた.
同じグループに所属するニューロンは発火確率を同じ時間帯にあげることで表現していた.
用いられていた手法では,ニューロンが複数グループに所属する場合やグループに所属しない場合も対応ができる.
最も良いと結論づけられていた手法はICA-CSとSGCという手法だったが,どちらの手法も閾値などのパラメータを決めなければならない.

Ghandorらは学習に関係するニューロン(engram cell)群についてNMFを用いて解析を行った\cite{Ghandour2019}.
実験は複数セッションに分けて行われており,それぞれのセッションでNMFを用いてニューロングループとその活動に分解していた.
基底数はAICcで決めていた.
セッションごとに推定されたグループが近いかをcosine similarityで計った結果,engram cellではnon-engram cellよりも繰り返し活動するグループが多かった.
NMFによってニューロングループがどれほど抽出できるかは論じられていなかった.

\section{目的}
カルシウムイメージングデータは上述のように多くのニューロンを観測できる利点がある一方,時間分解能が低いという欠点を持つ.
本研究で扱うデータは低いサンプリングレートで観測されたデータである.
ニューロンは複数で活動することで情報伝達を行っており,睡眠時に多数のニューロンが同時に活動する現象も確認されている.
これより,本データでは同時に活動するニューロングループを推定して解析を行うことが適当である.

本研究の目的は,カルシウムイメージングデータから同時に活動するニューロングループを推定することである.
紹介した関連研究は,本研究で扱いたいデータのサンプリングレートでは使えないものや,パラメータを適切に決めなければならないものだった.
提案アプローチでは,数理モデルに基づいてNMFを用いてニューロンの隣接行列を推定し,スペクトラルクラスタリングを用いて隣接行列からクラスタリングを行う.
隣接行列はバギングを用いて,2つのニューロンが同じグループになる確率値をつける.
そのため,グループに所属しないニューロンの推定や,どれくらい結果を信用して良いのかがわかる.
その際も閾値を決めなければいけないが,確率値という分かりやすい指標のため,既存手法より簡単だと思われる.
NMF自体も基底数を事前に決めなければいけないが,大体の基底数を決めればその基底数周りの結果を平均することでパラメータによる結果の劇的な変化を抑えることができる.
また,人工データ実験によって提案アプローチがカルシウムイメージングデータからニューロングループの情報をどの程度取り出せるかを確かめる.

本論文の構成は以下の通りである.
第2章では数理モデルを元にしたクラスタリングのアプローチを説明する.
第3章では人工データの作成方法と人工データ実験の結果を述べる.
第4章では実データ解析の結果を述べる.
第5章では採用に至らなかった検討事項について述べ,第6章に本研究をまとめる.
