\chapter{結論}
本論文では,サンプリングレートの低いカルシウムイメージングデータからニューロンのクラスタリングを行うことを目的に,バギングとNMFを用いた隣接行列推定がクラスタリングに有効であることを確かめた.
データの生成モデルを立てた上で,ニューロン同士の結合推定にNMFを用い,バギングによって結合確率を推定した.
NMFでは基底数を決めることが難しいが,モデル平均によってその問題を緩和した.
推定した隣接行列はスペクトラルクラスタリングを用いてクラスタリングを行った.
クラスタ数はNMFの基底数とは別に固有値ギャップから推定した.
人工データ実験の結果,真の基底数周りで隣接行列を推定できれば,ランダムに活動するニューロンがある場合やニューロンがランダムに活動する時間が多い場合に相互相関行列よりもロバストにグループ推定できることを確かめた.
また,グループに所属しないニューロンもある程度推定できることも確かめられた.
提案アプローチで実データ解析を行った結果,NREM睡眠の多いデータで大規模クラスタが推定された.
相互相関行列を用いたクラスタリングでは,パラメータによっては睡眠中の大規模クラスタが推定された.
しかし,結果はパラメータによって大きく変化したため,提案アプローチの方が解析方法として扱いやすいのではないかと思われる.

今後の展望としては,ニューロンが2つのグループに所属する場合に提案アプローチを拡張できるかを確かめる必要がある.
また,付録Bで述べる方法でNMFの真の自由度が測れるかどうかも実験を通して確かめたい面白い問題である.
