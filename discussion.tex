\chapter{結論}
本論文では,サンプリングレートの低いカルシウムイメージングデータからニューロンのクラスタリングを行うことを目的に,ブートストラップ法とNMFを用いた隣接行列推定がクラスタリングに有効であることを確かめた.
データの生成モデルを立てた上で,ニューロン同士の結合推定にNMFを用い,ブートストラップ法をによって結合確率を推定した.
NMFでは基底数を決めることが難しいが,モデル平均によってその問題を緩和した.
提案アプローチが有効であることを人工データ実験によって確かめた.

今後の展望としては,ニューロンが2つのグループに所属する場合にfuzzy clusteringなどが行えるかを確かめる必要がある.
また,第5章で述べた方法でNMFの真の自由度が測れるかどうかも実験を通して確かめたい面白い問題である.
