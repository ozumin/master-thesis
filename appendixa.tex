\chapter{付録A 人工データの作成方法}
ニューロン集団のカルシウムイメージングデータをシミュレーションによって作り,解析手法を評価する.
シミュレーションでは1)ニューロンのネットワーク構造を作成し,2)スパイクのシミュレーションを行い,3)蛍光強度の観測データに変換する.
\section{ネットワーク構造}
シミュレーションに用いるニューロンの個数を$I$として,ニューロンのネットワーク構造を$S \in \{0, 1\}^{I \times I}$とする.
$s_{ij}$はニューロン$i$からニューロン$j$へ活動電位が伝わるかを表している.
本節では$S$の作り方を説明する.

ニューロンのネットワーク構造にはsmall world network\cite{Watts1998}を用いる.
Small world networkはノード数,張り替え確率,初期次数を決めることによってネットワークを作成するアルゴリズムである.
初期次数は,ニューロンが平均何個のニューロンとシナプス結合を持つかという変数である.
張り替え確率は,初期次数によって作成された規則的なグラフのエッジをランダムに張り替える確率である.
そのため,エッジのうち何割が遠くのニューロンとつながっているかを表す変数である.

実際のニューロンをsmall world networkによって表すために,初期次数と張り替え確率を実データから決める.
今回はこの値はニューロンのコネクションの割合と相互のコネクションの割合から決める.
興奮性ニューロン同士の6.7\%であり,そのうち双方向のコネクションの割合は24\%である\cite{Jouhanneau2015}.
発達中マウスの興奮性ニューロンから抑制性ニューロンへのコネクティビティと抑制性ニューロンから興奮性ニューロンへのコネクティビティはどちらも78\%であった\cite{Holmgren2003}.
成熟したマウスではより少ないと思われるが,データが見つからなかったため,40\%とした.
相互のコネクションの割合がランダムにエッジを作るよりも高いのは,近いニューロンにコネクションが作られやすいからだと考えられる.
これらのデータを実現するように初期次数と張り替え確率を調整した.
用いたパラメータを\Tabref{tab:parameter1}に示す.
抑制性ニューロン同士のコネクティビティは分からないため,興奮性ニューロンと同じにしている.

\begin{table}[htb]
  \center
  \begin{tabular}{|c|cc|} \hline
    結合の種類 & 初期次数 & 張り替え確率 \\ \hline
		同種類のニューロン間 & $0.0335 N$ & $0.3$ \\
		興奮性ニューロンと抑制性ニューロン間 & $0.2I$ & $0.3$\\ \hline
  \end{tabular}
  \caption{ネットワーク構造のパラメータ値}
  \label{tab:parameter1}
\end{table}

実際のネットワーク構造の作り方を説明する.
ネットワーク構造は興奮性ニューロン同士の結合,抑制性ニューロン同士の結合,興奮性ニューロンと抑制性ニューロン間の結合の3つに分けて作成する.
まず,全ニューロンのうち抑制性ニューロンと興奮性ニューロンのインデックスを決めておく.
全てのニューロンについて\Tabref{tab:parameter1}に従ってネットワークを作成し,それぞれに対応する隣接行列の上三角または下三角行列を取り出して結合する.
作成したいのは向きのある有向グラフなので,上三角行列と下三角行列を分けて作成する.

\section{スパイクシミュレーション}
スパイクのシミュレーションにIzhikevichモデル~\cite{Izhikevich2003}を用いる.
このモデルはHodgikin-Huxleyモデルをもとにしており,計算コストが低い.
Izhikevichモデルでは,あるニューロンの膜電位が閾値を超えると発火したとみなし,あらかじめ定義したニューロンのネットワーク構造に従って結合を持つニューロンの膜電位を上昇させる.
このシミュレーションで設定しなければいけないのは,個々のニューロンの特徴パラメータ,重み付きのネットワーク構造,外部からのランダムな入力である.

まず,個々のニューロンの特徴パラメータについて説明する.
このモデルではニューロンごとに4つのパラメータを設定する必要があり,そのパラメータでニューロンを特徴づける.
本論文では興奮性ニューロンにはregular spiking neurons,抑制性ニューロンにはfast spiking neuronsを用いる.
それらのパラメータを~\Tabref{tab:parameter2}に示す.
ただし,$r_e$と$r_i$は0から1の一様分布に従う確率変数である.

\begin{table}[htb]
  \center
  \begin{tabular}{|c|cccc|} \hline
    ニューロンの種類 & a & b & c & d \\ \hline
    興奮性ニューロン & 0.02 & 0.2 & $-65 + 15 r_e^2$ & $8 - 6r_e^2$ \\
    抑制性ニューロン & $0.02 + 0.08r_i$ & $0.25 - 0.05 r_i$ & -65 & 2 \\ \hline
  \end{tabular}
  \caption{Izhikevichモデルのパラメータ値}
  \label{tab:parameter2}
\end{table}

次に,重み付きのネットワーク構造$W \in \mathbb{R}^{I \times I}$について説明する.
ニューロン$i$から$j$へ結合があった場合,$w_{ij}$はニューロン$i$が発火した時にニューロン$j$の膜電位をどれだけ上昇させるかという数値である.
$W$は,前節で作成した$S$の非ゼロ要素を数値で置き換えることで作成する.
その際,同じグループへの興奮性ニューロンからの入力は強めにする.
同じグループの興奮性ニューロンからの結合は一様分布$U(7,10)$,異なるグループの興奮性ニューロンからの結合は標準偏差$\sigma_w = 1.5$の対数正規分布の$7$以下の分布,抑制性ニューロンからの結合は一様分布$U(-10,0)$からサンプルする.
対数正規分布を用いる理由は\cite{Song}のデータに基づく.
\begin{align}
	w_{ij} = \begin{cases}
		U(7,10) & (s_{ij} = 1 \text{,$i$は興奮性ニューロン,$i$と$j$は同じグループ}) \\
		\{\frac{1}{\sqrt{2 \pi \sigma_w^2}w} \exp \{ - \frac{(\ln w)^2}{2 \sigma_w^2} | w \in [0,7]\} & (s_{ij} = 1 \text{,$i$は興奮性ニューロン,$i$と$j$は異なるグループ}) \\
		U(-10,0) & (s_{ij} = 1 \text{$i$は抑制性ニューロン}) \\
		0 & (s_{ij} = 0)
  \end{cases}
	\label{eq:W}
\end{align}

最後に外部からのランダムな入力について説明する.
ニューロンには観測範囲外からの入力がある(以降,外部入力とする).
そのため,シミュレーション中も外部からの電位を乱数としてニューロンの電位に足す.
本論文では,ニューロンの活動も外部入力の大きさで表現する.
活動していない興奮性ニューロンと抑制性ニューロンにはそれぞれ,$\mathcal{N}(0,9)$と$\mathcal{N}(0,0.01)$に従う乱数を足す.
活動している興奮性ニューロンと抑制性ニューロンにはそれぞれ,$\mathcal{N}(0.5, 9)$と$\mathcal{N}(0.2,0.01)$に従う乱数を足す.
これらを\Tabref{tab:parameter3}に示す.
活動していないニューロンへの外部入力は\cite{Izhikevich2003}で用いられていたものを採用した.
ただし,興奮性ニューロンの活動時の外部入力は変化させた実験もある.

\begin{table}[htb]
  \center
  \begin{tabular}{|c|cc|} \hline
    ニューロンの種類 & 活動時の外部入力 & 活動していない時の外部入力 \\ \hline
		興奮性ニューロン & $\mathcal{N}(0.5,9)$ & $\mathcal{N}(0, 9)$ \\
		抑制性ニューロン & $\mathcal{N}(0.2, 0.01)$ & $\mathcal{N}(0, 0.01)$ \\ \hline
  \end{tabular}
  \caption{シミュレーションに用いる外部入力の値}
  \label{tab:parameter3}
\end{table}

実際の脳でもこのように外部からの入力によってニューロンの活動を制御していると考えられる.
あるニューロングループを活動させる別の方法として,そのグループのハブとなるニューロンにのみ強い外部入力を与える方法も考えられる.

実際にマウスのニューロンの発火頻度がどれくらいなのか\cite{Watson2016}を元に\Tabref{tab:spike-frequency}に示す.
\begin{table}[htb]
  \center
  \begin{tabular}{|c|ccc|} \hline
    ニューロンの種類 & 覚醒時(Hz) & ノンレム睡眠時(Hz) & レム睡眠時(Hz) \\ \hline
		興奮性ニューロン & $0.76 \pm 1.53$ & $0.69 \pm 0.86$ & $0.88 \pm 1.33$ \\
		抑制性ニューロン & $5.59 \pm 7.25$ & $4.69 \pm 5.62$ & $4.25 \pm 9.43$ \\ \hline
  \end{tabular}
  \caption{ニューロンごとの発火頻度の中央値}
  \label{tab:spike-frequency}
\end{table}

\section{カルシウムイメージングモデル}
スパイクデータからカルシウムイオン濃度を計算する~\cite{Vogelstein2009}のモデルを用いる:
\begin{equation}
  [\text{Ca}^{2+}]_{i,t} - [\text{Ca}^{2+}]_{i,t-1} = - \frac{\Delta}{\tau}([\text{Ca}^{2+}]_{i,t-1} - [\text{Ca}^{2+}]_b) + An_{i,t} + \sigma_c \sqrt{\Delta} \epsilon_{i,t},
  \label{eq:calcium}
\end{equation}
ただし,$[\text{Ca}^{2+}]_{i,t}$をニューロン$i$の時刻$t$でのカルシウムイオン濃度,$[\text{Ca}^{2+}]_b$をカルシウムイオン濃度のベースライン,$\Delta$を時間幅,$\tau$は時定数,$A$は1つのスパイクでのカルシウムイオン濃度の上がり幅,$n_{i,t} \in \{0,1\}$はニューロン$i$の時刻$t$でのスパイク,$\sigma_c$はノイズの分散,$\epsilon_{i,t}$は標準正規分布に従う確率変数である.
この人工データでは,蛍光強度がある値から上昇しない飽和状態は考えないこととする.

次に,同論文のモデルを使ってカルシウムイオン濃度$[\text{Ca}^{2+}]_{i,t}$をカルシウムイメージングで計測される蛍光強度$F_{i,t}$に変換する:
\begin{equation}
	F_{i,t} = \alpha[\text{Ca}^{2+}]_{i,t} + \beta + \sigma_F \epsilon_{i,t},
  \label{eq:intensity}
\end{equation}
$\alpha$は強度,$\beta$はバイアス,$\sigma_F$はノイズの分散である.
\Tabref{tab:parameter2}に使用したパラメータを示す.
何種類かの蛍光タンパク質の性能を調べた論文に\cite{Chen2013a}がある.
この論文の,1秒間に10回発火した時のdecay time(蛍光強度が上がり切ってから半分の強度になるまでの時刻)から$\tau = 2.3$とし,SN比から$\sigma_c = 0.5$とした.

\begin{table}[htb]
  \center
  \begin{tabular}{c|c}
		\multicolumn{2}{c}{パラメータ値} \\ \hline
		$[\text{Ca}^{2+}]_b$ & 0.1\\
		$\Delta$ & 0.001\\
		$\tau$ & 2.3\\
		$A$ & 5.0\\
		$\sigma_c$ & 0.5 \\
		$\alpha$ & 1.0\\
		$\beta$ & 10\\
		$\sigma_F$ & 1.0\\ \hline
  \end{tabular}
  \caption{カルシウムイメージングモデルでのパラメータ値}
  \label{tab:parameter2}
\end{table}

\section{観測モデル}
実データは8[Hz]でサンプリングされたデータなので,シミュレーションした蛍光強度を125[ms]ごとに足し合わせる:
\begin{equation}
  x_{i,t'} = \sum_{t=1}^{125} F_{i,t},
  \label{eq:observation}
\end{equation}
ここで,$t'$はサンプリング後の時刻を表す.
