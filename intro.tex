\chapter{序論}
% 睡眠は脳によって制御されており\cite{Hobson2005},哺乳類にとって必要不可欠な生理現象である.
% その重要性にも関わらず,睡眠について解明されていないことが多い.
% その中でも,睡眠とはいかなる生物学的な状態か,という問いに対する明確な答えは未だない\cite{Kanda2016}.
% 
% 哺乳類の睡眠状態は脳波によって定義される.
% しかし,哺乳類以外は脳波を計測することができないためふるまいでしか評価できない.
% そこで,ニューロンの活動から睡眠を新たに定義することができれば睡眠状態の解明に繋がると考えられる\cite{Kanda2020}.
% 
% 脳内の情報伝達は複数個のニューロンによって行われている.
% また,睡眠時には多数のニューロンが活動してある現象が見られることが知られている.
% 複数ニューロンの活動を解析することが重要である.
脳の機能解明のためにニューロンの活動解析が行われている.
ニューロンは複数で活動することによって情報処理を行っているため,集団に注目して解析することが重要である.
ニューロンの観察方法として,パッチクランプ法,細胞内記録法,細胞外記録法などの電気生理学的な手法が挙げられる.
これらの手法は十分な時間分解能かつ細胞レベルでニューロンを観察することができる.
しかし,電気生理学的な手法では電極から近いニューロンしか電位変化の計測できないため,観察できるニューロンの数は数十から多くても数百程度である.

より多くのニューロンを観察するために,蛍光イメージングの1つであるカルシウムイメージングという手法が用いられる.
ニューロンで活動電位が発生(発火)すると細胞内のCa${}^{2+}$濃度が上昇する.
カルシウムイメージングでは,このCa${}^{2+}$濃度上昇を蛍光で可視化する.
具体的には,Ca${}^{2+}$と結合すると蛍光強度が変化する蛍光分子を細胞内に発現させておき,Ca${}^{2+}$濃度を蛍光強度として蛍光顕微鏡で観察する.
蛍光イメージングを用いる利点として,(1)高い空間分解能,(2)広い観察範囲,(3)遺伝子工学と併用して興奮性/抑制性ニューロンの同定などをした上での観察ができることが挙げられる.
一方,時間分解能が電気生理学的手法よりも低いことが蛍光イメージングの欠点である.
Ca${}^{2+}$濃度の変化はニューロンの電気的変化よりも遅く,また,カルシウム感受性蛍光分子のキネティクスも影響する.
さらに,カメラやレーザースキャンでのサンプリングレートは高くても100[Hz]程度であり,ニューロンの個々の発火を全て捉えるには不十分である.
これより,カルシウムイメージングデータからはニューロン活動の相関しか取ることができないと考えられる.

ニューロン集団の解析にはクラスタリングやネットワーク構造推定などが考えられる.
相関しか取れないカルシウムイメージングデータからは向きのある有向グラフの推定は難しく,向きのない無向グラフの推定しかできない.
クラスタリングは,同じグループ内のニューロン同士がエッジを持つ無向グラフとしても考えられる.
結果の解釈はクラスタリングの方が行いやすい.
ニューロンの集団解析で考えられる最も単純な方法は,相互相関行列を用いてクラスタリングを行う方法である.
しかし,ニューロンが同時活動する時間が短い場合には同じグループ内のニューロンの相関がグループ外のニューロンと比べて高くならず,意味のある情報が抽出できないと思われる.
また,どのグループにも所属しないニューロンの存在も考えられるが,相互相関行列ではこのようなニューロンの判定は難しい.

本研究では,低い時間分解能のカルシウムイメージングデータからニューロンをクラスタリングするアプローチを提案し,人工データ実験を通してどの程度の情報が抽出できるかを確認する.
提案アプローチでは,ニューロンが同じグループに所属する確率値を無向グラフの隣接行列として推定し,それをスペクトラルクラスタリングによってクラスタリングする.
グループに所属しないニューロンは,他のニューロンとグループになる確率が低いものとして推定することができる.
隣接行列の推定には数理モデルに基づいてnon-negative matrix factorization(NMF)を用いることで,相互相関行列や他の行列分解手法よりも精度の良いクラスタリングができると考えられる.
