\chapter{序章}
\section{背景}
睡眠は脳によって制御されており\cite{Hobson2005},哺乳類にとって必要不可欠な生理現象である.
その重要性にも関わらず,睡眠については解明されていないことが多い.
その中でも睡眠とはいかなる生物学的な状態か,という問いに対する明確な答えは未だない\cite{Kanda2016}.

睡眠状態にあることは,哺乳類は脳波によって定義される,
しかし,哺乳類以外は脳波を計測することができないためふるまいによって定義される.
そこで,ニューロンの活動から睡眠を新たに定義することができれば睡眠状態の解明に繋がると考えられる\cite{Kanda2020}.

ニューロンの観察方法として,パッチクランプ法,細胞内記録法,細胞外記録法などの電気生理学的な手法が挙げられる.
これらの手法は十分な時間分解能かつ細胞レベルでニューロンを観察することができる.
しかし,電気生理学的な手法では観察できるニューロンの数は数十から多くても数百程度である.

より多くのニューロンを観察するために,蛍光イメージングの1つであるカルシウムイメージングという手法が用いられる.
ニューロンで活動電位が発生(発火)すると細胞内の$\text Ca^{2+}$濃度が上昇する.
カルシウムイメージングでは,この$\text Ca^{2+}$濃度上昇を蛍光で可視化する.
具体的には,$\text Ca^{2+}$と結合すると蛍光強度が変化する蛍光分子を細胞内に発現させておき,$\text Ca^{2+}$濃度を蛍光強度として蛍光顕微鏡で観察する.
蛍光イメージングを用いる利点として,(1)高い空間分解能,(2)広い観察範囲,(3)遺伝子工学と併用して興奮性/抑制性ニューロンの同定などをした上での観察ができることが挙げられる.
一方,時間分解能が電気生理学的手法よりも低いことが蛍光イメージングの欠点である.
$\text Ca^{2+}$濃度の変化はニューロンの電気的変化よりも遅く,また,カルシウム感受性蛍光分子のキネティクスも影響する.
さらに,カメラやレーザースキャンでのサンプリングレートは高くても100Hz程度であり,ニューロンの個々の発火を全て捉えるには不十分である.

\section{目的}
本研究の目的は,同時に活動するニューロングループを推定し,それらの睡眠・覚醒時の活動の違いを解析することである.
そのためにカルシウムイメージングデータに合った解析のアプローチを模索する.

本論文の構成は以下の通りである.
